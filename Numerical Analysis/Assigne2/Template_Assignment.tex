%Tutorial template
\documentclass[a4paper,12pt]{article}
\usepackage[utf8]{inputenc}
\usepackage{amsmath,amssymb}
\usepackage[left=2cm,right=2cm,top=2cm,bottom=2cm]{geometry}

%Definition of variables
\newcommand{\AIMS}{AIMS-Senegal}
\newcommand{\Professor}{Dr SANGHARE}
\newcommand{\Course}{Numerical Analysis}
\newcommand{\Type}{Assignment 2 }
\newcommand{\Title}{Problem Solving No 2}
\newcommand{\Code}{Homework}
\newcommand{\Due}{\today}
\newcommand{\Student}{}
\newcommand{\AID}{}

%New Commands
\newcommand{\Createtitle}{
\vspace*{-4mm}
\begin{center}
\begin{Huge}
\textbf{\Title}\vspace{5mm}\\
\end{Huge}
\end{center}
}

%FancyHDR
\usepackage{fancyhdr}
\pagestyle{fancy}
\usepackage{lastpage}
\renewcommand\headrulewidth{1pt}
\fancyhead[L]{{\headheight = 18pt}{\scriptsize{\Professor}\vspace{-3pt}\newline \Course}}
\fancyhead[C]{{\headheight = 18pt}{\scriptsize{\Type}\vspace{-3pt}\\ \Title}}
\fancyhead[R]{{\scriptsize \Student \textit{\AID}}\Code}
\renewcommand\footrulewidth{1pt}
\fancyfoot[L]{{\scriptsize \AIMS}}
\fancyfoot[C]{\textbf{Page \thepage/\pageref{LastPage}}}
\fancyfoot[R]{\Due}

\begin{document}
\Createtitle

Name of the students in this assignemnt\footnote{Don't write the name of any student that didn't work with the group, cheating is stricly forbidden}
\begin{enumerate}
\item Student 1  El Hadji NGOM No 1637
\item Student 2  Mamadou Lamine THIAM No 1640
\item Student 3  Djibril GUEYE No 1629
\end{enumerate}

\section{Exercice 1:}

Soit la fonction \[f(x)=\frac{X-1}{2x+1}\]Soient $X_{1}=-1$; $X_{2}=0; X_{3}=1$ \\
\begin{enumerate}
\item Construction du polynôme de Lagrange:\\
$P(X)= \displaystyle\sum_{i=1}^{n+1}l_{i}(X)Y_{i}$ avec ${l_{i}}(x)=\displaystyle\prod_{j=1,j\neq i}^{n+1} \frac{X-X_{j}}{X_{i}-X_{j}}$.\\

Calcul des différentes valeurs de Y:\\
$Y_{1}=f(-1)=2$; $Y_{2}=f(X_{2})=-1$; $Y_{3}=f(X_{3})=0$ \\

Cherchons les différentes $l_{i}$:\\
$l_{1}(X)=(\frac{X-X_{2}}{X_{1}-X_{2}})(\frac{X-X_{3}}{X_{1}-X_{3}})$\\
$l_{1}(X)=\frac{X^{2}-X}{2}$ \\ 

$l_{2}(x)=(\frac{X-X_{1}}{X_{2}-X_{1}})(\frac{X-X_{3}}{X_{2}-X_{3}})$\\
$l_{2}(x)= -X^{2}+1$ \\

$l_{3}(x)=(\frac{X-X_{1}}{X_{3}-X_{1}})(\frac{X-X_{2}}{X_{3}-X_{2}})$\\
$l_{3}(x)= (\frac{X+1}{2})(X)=\frac{X^{2}+X}{2}$ \\

Le polynôme de Lagrange est donnée par:\\

\[P(X)=l_{1}(X)Y_{1}+l_{2}(X)Y_{2}+l_{3}(x)Y_{3}\]
$P(X)=2(\frac{X^{2}-X}{2})-(-X^{2}+1)$ \\
$P(X)= X^{2}-X+X^{2}-1 $
  \[\boxed{P(X)=2X^{2}-X-1}\] \\

\item  le polynôme de Newton s'écrit \\ $P(X)=\lambda_{1}+\lambda_{2}(X-X_{1})+\lambda_{3}(X-X_{1})(X-X_{2})$ \\
 
$\lambda_{1}=Y_{1}=2$\\

$\lambda_{2}=Y[X_{1},X_{2}]=\frac{Y[X_{2}]-Y[X_{1}]}{X_{2}-X_{1}}$\\
$\lambda_{2}=\frac{Y_{2}-Y_{1}}{X_{2}-X_{1}}=\frac{-1-2}{0+1}=-3$ \\

$\lambda_{3}=Y[X_{1},X_{2},X_{3}]$\\

$\lambda_{3}=\frac{Y[X_{2},X_{3}]-Y[X_{1},X_{2}]}{X_{3},X_{1}}$\\
 
$\lambda_{3}=\frac{\frac{Y_{3}-Y_{2}}{X_{3}-X_{2}}-\frac{Y_{2}-Y_{1}}{X_{2}-X_{1}}}{X_{3}-X_{1}}$\\

$\lambda_{3}=\frac{\frac{0+1}{1-0}-\frac{-3}{0+1}}{1+1}=2$\\ 

Le polynôme de Newton est donné par: \\
$P(X)=\lambda_{1}+\lambda_{2}(X-X_{1})+\lambda_{3}(X-X_{1})(X-X_{2})$
$P(X)=2-3(X+1)+2(X+1)(X)$ \\ 
\[\boxed{P(X)=2X^{2}-X-1}\]
\item La droite de régression au sens des moindres carrés des points $(-1,2), (0,-1),(1,0)$\\

Une équation de la droite de régression par la méthodes moindres carrés est: \[y=ax+b\] Avec $a=\frac{\displaystyle\sum_{i=0}^n (X_{i}-\overline{X})(Y_{i}-\overline{Y})}{\displaystyle\sum_{i=0}^n (X_{i}-\overline{X})^{2}}$ et $b=\overline{Y}-a\overline{X}$\\

Calcul des moyennes $\overline{X}$ et $\overline{Y}$:\\

$\overline{X}=\frac{1}{3}(-1+0+1)=0$\\$\overline{Y}=\frac{1}{3}(+2-1+0)=\frac{1}{3}$\\

$a=\frac{(X_{1}-\overline{X})(Y_{1}-\overline{Y})+(X_{2}-\overline{X})(Y_{2}-\overline{Y})+(X_{3}-\overline{X})(Y_{3}-\overline{Y})}{(X_{1}-\overline{X})^{2}+(X_{2}-\overline{X})^{2}+(X_{3}-\overline{X})^{2}}$

$a=\frac{(-1)(2-\frac{1}{3})+(0)(-1-\frac{1}{3})+(1)(-\frac{1}{3})}{(-1)^{2}+(0)^{2}+(1)^{2}}=-1$ \\
On trouve:
$\boxed{a=-1}$ et $\boxed{b=\frac{1}{3}}$ \\ Ainsi l'équation est donnée par: $\boxed{y=-x+\frac{1}{3}}$

\end{enumerate}







\section{\textbf{Exercise 2}}
\begin{itemize}
\item[\textbf{a.)}] The Exact Solution
\begin{align*}
f\left( x\right) = x^{3} - 3x^{2} + 2\\
I &=  \int_{1}^{4} = f\left( x\right) dx\\
&= \int_{1}^{4}\left(   x^{3} - 3x^{2} + 2\right)  dx\\
&= \left[ \frac{1}{4} x^{4} - x^{3} + 2x \right]_{1}^{4} \\
&= \left( \dfrac{1}{4}\left( 4\right)^{4} - \left( 4\right)^{3} +  2\left( 4\right) \right) - \left( \dfrac{1}{4}\left( 1\right)^{4} - \left( 1\right)^{3} +  2\left( 1\right) \right)\\  
&= \left( \dfrac{256}{4}-64 +8 \right) -  \left( \dfrac{1}{4}- 1 + 2 \right)\\
&= \left( 64 - 64 + 8 \right) - \left( \frac{1}{4} + 1\right) \\ 
&= \frac{28-1}{4}\\
&= \frac{27}{4}
\end{align*}
\item[b.] By subdividing the interval $[1,4]$
 into $3$ intervals.\\
$$f\left( x\right) = x^{3} - 3x^{2} + 2$$
\\
\begin{align*}
x_1&=1; & x_2&=2; & x_3&=3; & x_4&=4\\
\end{align*}

\item[\textbf{i.)}] Using the Trapezium method
\begin{align*}
I &=  \int_{a}^{b} f\left( x\right) dx = \left(f \left(a \right)+f\left( b\right) \right) * \frac{b-a}{2}\\
  &= \sum_{k=1}^{N}  \int_{a}^{b} f\left( x\right) dx \approx \dfrac{h}{2} \left(f(x(k)) + f(x(k+1))\right)\\
  h &= \frac{\left( x_{n} - x_{0}\right)}{N} = \dfrac{4-1}{3} = 1\\
f\left( x_{1}\right) &= 1^{3} - 3\left( 1\right)^{2} + 2 = 1 - 3 + 2 = 0\\
f\left( x_{2}\right) &= 2^{3} - 3\left( 2\right)^{2} + 2 = 8 - 12 + 2 = -2\\
f\left( x_{3}\right) &= 3^{3} - 3\left( 3\right)^{2} + 2 = 27 - 27 + 2 = 2\\
f\left( x_{4}\right) &= 4^{3} - 3\left( 4\right)^{2} + 2 = 64 - 48 + 2 = 18\\
k = 1\\
\frac{h}{2}\left[ f\left( x_{k}\right)    + f\left( x_{k+1}\right) \right] &= \frac{1}{2}\left[ f\left( x_{1}\right) + f\left( x_{2}\right) \right] = \frac{1}{2}\left( 0 - 2 \right) = -1 \\
k = 2\\
\frac{h}{2}\left[ f\left( x_{k}\right)    + f\left( x_{k+1}\right) \right] &= \frac{1}{2}\left[ f\left( x_{2}\right) + f\left( x_{3}\right) \right] = \frac{1}{2}\left( -2 + 2 \right) = 0 \\
k=3\\
\frac{h}{2} \big[ f(x_k)+ f(x(k+1))\big]&=\frac{1}{2}\big[ f(x_3)+f(x_4)\big]=\frac{1}{2}\big[ 2+18\big]=10\\
\int^4_1 f(x)dx &= -1+0+10=9
\end{align*}
\item[ii.] Using the Simpson's method
\begin{align*}
\int_a^b f(x)dx&=\sum \limits_{k=1}^{4} \frac{h}{6} \big [f(x_{k}) + 4f \big(\frac{x_{k}+x_{k+1}}{2}\big)+f(x_{k+1})\big]\\
f\left( x_{1}\right) &= 1^{3} - 3\left( 1\right)^{2} + 2 = 1 - 3 + 2 = 0\\
f\left( x_{2}\right) &= 2^{3} - 3\left( 2\right)^{2} + 2 = 8 - 12 + 2 = -2\\
f\left( x_{3}\right) &= 3^{3} - 3\left( 3\right)^{2} + 2 = 27 - 27 + 2 = 2\\
f\left( x_{4}\right) &= 4^{3} - 3\left( 4\right)^{2} + 2 = 64 - 48 + 2 = 18\\
f(\frac{x_1+x_2}{2})&=f(\frac{1+2}{2})=f(\frac{3}{2})=\frac{27}{8}-3\big(\frac{9}{4}\big)+2=\frac{27}{8}-\frac{27}{4}+2=\frac{-11}{8}\\
f(\frac{x_2+x_3}{2})&=f(\frac{2+3}{2})=f(\frac{5}{2})=\frac{125}{8}-3\big(\frac{25}{4}\big)+2=\frac{125}{8}-\frac{75}{4}+2=\frac{-9}{8}\\
f(\frac{x_3+x_4}{2})&=f(\frac{3+4}{2})=f(\frac{7}{2})=\frac{343}{8}-3\big(\frac{49}{4}\big)+2=\frac{343}{8}-\frac{147}{4}+2=\frac{65}{8}\\
k=1\\
\frac{1}{6} \big[ f(x_1)+4f \big(\frac{x_1+x_2}{2}\big)+f(x_2)\big]&=\frac{1}{6}\big[ 0+4(\frac{-11}{2})-2\big]=\frac{1}{6}\big[\frac{-11}{2}-2 \big]=\frac{-15}{12}\\
k=2\\
\frac{1}{6} \big[ f(x_2)+4f \big(\frac{x_2+x_3}{2}\big)+f(x_3)\big]&=\frac{1}{6}\big[ -2+4(\frac{-9}{8})+2\big]=\frac{1}{6}\big[\frac{-9}{2} \big]=\frac{-9}{12}\\
k=3\\
\frac{1}{6} \big[ f(x_3)+4f \big(\frac{x_3+x_4}{2}\big)+f(x_4)\big]&=\frac{1}{6}\big[ 2+4(\frac{65}{8})+18\big]=\frac{1}{6}\big[20+\frac{65}{2} \big]=\frac{105}{12}\\
\int^4_1 f(x)dx&=\frac{-15}{12}-\frac{9}{12}+\frac{105}{12}=\frac{81}{12}=\frac{27}{4}
\end{align*}

\item[\textbf{3.)}] Comparing the results with the exact value.\\
\begin{align*}
for \ Trapezium \ method\\
Error &= \left| \dfrac{I_{trap} - I_{exa}}{Itrap}\right|\times100 = \left|\dfrac{9 - \dfrac{27}{4}}{9}\right|\times 100\\
 &= \dfrac{1}{4}\times 100 = 25\%\\
for \ Simpson \ method\\
Error &= \left| \dfrac{I_{trap} - I_{exa}}{Itrap}\right|\times100 = \left|\dfrac{\dfrac{27}{4} - \dfrac{27}{4}}{\dfrac{27}{4}}\right|\times 100\\
 &= 0 \times 100 = 0\%
\end{align*}
Hence, the Simpson method is more accurate than the Trapezoid method.
\end{itemize}
Programme scinotes (Exercise 2)

\section{Exercise 3}
Voir Programme fichier scinotes
\section{Exercice 4}
Soit 
$$A =\left(
\begin{array}{ccc}
 4&2&1\\
 -1&2&0\\
 2&1&4\\
\end{array}
\right)
\hspace{1cm}
b=\left(
\begin{array}{c}
 4\\
 2\\
 9\\
\end{array}
\right)
\hspace{1cm}
X_0 = \left(
\begin{array}{c}
 4\\
 2\\
 9\\
\end{array}
\right)
$$
Resolvons $Ax=b$ par la m\'ethode de:\\\underline{Jacobi:}
$$
X_1 \left\{
\begin{array}{ccccc}
 x_1^1 & = & \frac{1}{4}(4-2\times0 - 0 )& = &1 \\
 x_1^2 & = & \frac{1}{2}(2+1\times0 - 0 )& = &1 \\
 x_1^3 & = & \frac{1}{4}(9-2\times0 -1\times 0 )& =& \frac{9}{4} \\
\end{array}
\right.
$$

\[
X_2 \left\{
\begin{array}{ccccc}
 x_2^1 & = & \frac{1}{4}(4-2\times1 -\frac{9}{4} )& = &-\frac{1}{16} \\
 x_2^2 & = & \frac{1}{2}(2+1\times1 - 0 )& = &\frac{3}{2} \\
 x_3^3 & = & \frac{1}{4}(9-2\times1 -1\times 1 )& =& \frac{3}{2} \\
\end{array}
\right.
\]

$$
X_3 \left\{
\begin{array}{ccccc}
 x_3^1 & = & \frac{1}{4}(4-2\times\frac{3}{2} - \frac{3}{2} )& = &-\frac{1}{8} \\
 x_3^2 & = & \frac{1}{2}(2+1\times\frac{-1}{16} - 0 )& = &\frac{31}{32} \\
 x_3^3 & = & \frac{1}{4}(9-2\times\frac{-1}{16} -1\times\frac{3}{2} )& =& \frac{61}{32} \\
\end{array}
\right.
$$

$$
X_4 \left\{
\begin{array}{ccccc}
 x_4^1 & = & \frac{1}{4}(4-2\times\frac{31}{32} - \frac{61}{32} )& = &-\frac{5}{128} \\
 x_4^2 & = & \frac{1}{2}(2+1\times\frac{-1}{8} - 0 )& = &\frac{15}{16} \\
 x_4^3 & = & \frac{1}{4}(9-2\times\frac{-1}{8}-1\times\frac{31}{32})&=& \frac{265}{128}\\
\end{array}
\right.
$$

On obtient les vecteurs suivants apr\'es 4 itr\'erations:

$$
X_1=\left(
\begin{array}{c}
 1\\
 1\\
 \frac{9}{4}\\
\end{array}
\right)
\hspace{1cm}
X_2 = \left(
\begin{array}{c}
 -\frac{1}{16}\\
 \frac{3}{2}\\
 \frac{3}{2}\\
\end{array}
\right)
\hspace{1cm}
X_3 = \left(
\begin{array}{c}
 -\frac{1}{8}\\
 \frac{31}{32}\\
 \frac{61}{32}\\
\end{array}
\right)
\hspace{1cm}
X_4 = \left(
\begin{array}{c}
 \frac{5}{128}\\
 \frac{15}{16}\\
 \frac{265}{128}\\
\end{array}
\right)
$$
Ainsi on peut conclure la m\'ethode converge vers une solution 
$$
X = \left(
\begin{array}{c}
 0\\
 1\\
 2\\
\end{array}
\right)
$$
\underline{ Gauss-Seidel:}
$$
X_1 \left\{
\begin{array}{ccccc}
 x_1^1 & = & \frac{1}{4}(4-2\times0 - 0 )& = &1 \\
 x_1^2 & = & \frac{1}{2}(2+1\times0 - 0 )& = &1 \\
 x_1^3 & = & \frac{1}{4}(9-2\times0 -1\times 0 )& =& \frac{9}{4} \\
\end{array}
\right.
$$

$$
X_2 \left\{
\begin{array}{ccccc}
 x_2^1 & = & -\frac{1}{16} \\
 x_2^2 & = & \frac{1}{2}(2+1\times\frac{1}{16} - 0 )& = & 0.968 \\
 x_2^3 & = & \frac{1}{4}(9-2\times\frac{-1}{16} -0.968 )& =& 2.039\\
\end{array}
\right.
$$

$$
X_3 \left\{
\begin{array}{ccccc}
 x_3^1 & = & -\frac{1}{8} \\
 x_3^2 & = & \frac{1}{2}(2+1\times\frac{-1}{8} - 0 )& = & 0.937 \\
 x_3^3 & = & \frac{1}{4}(9-2\times\frac{-1}{8} -0.937 )& =& 1.953\\
\end{array}
\right.
$$

$$
X_4 \left\{
\begin{array}{ccccc}
 x_4^1 & = & \frac{5}{128} \\
 x_4^2 & = & \frac{1}{2}(2+\times\frac{5}{128} - 0 )& = & 2.039 \\
 x_4^3 & = & \frac{1}{4}(9-2\times\frac{5}{128} -2.039 )& =& 1.720\\
\end{array}
\right.
$$
On obtient les vecteurs suivants apr\'es 4 itr\'erations:

$$
X_1=\left(
\begin{array}{c}
 1\\
 1\\
 \frac{9}{4}\\
\end{array}
\right)
\hspace{1cm}
X_2 = \left(
\begin{array}{c}
 -\frac{1}{16}\\
 0.968\\
 2.039\\
\end{array}
\right)
\hspace{1cm}
X_3 = \left(
\begin{array}{c}
 -\frac{1}{8}\\
 0.937\\
 1.953\\
\end{array}
\right)
\hspace{1cm}
X_4 = \left(
\begin{array}{c}
 \frac{5}{128}\\
 2.039\\
 1.72\\
\end{array}
\right)
$$
Ainsi on peut conclure la m\'ethode converge vers une solution:
$$
X = \left(
\begin{array}{c}
 0\\
 1\\
 2\\
\end{array}
\right)
$$

\section{Exercise 5}









\end{document}