\documentclass[17pt,a4paper]{article}
\usepackage[utf8]{inputenc}
\usepackage{amsmath}
\usepackage{amsfonts}
\usepackage{amssymb}
\usepackage{amsthm}
\usepackage{amstext}
\usepackage{amsbsy}
\usepackage{amsxtra}

\begin{document}


\section{Exercice 1:}

Soit la fonction \[f(x)=\frac{X-1}{2x+1}\]Soient $X_{1}=-1$; $X_{2}=0; X_{3}=1$ \\
\begin{enumerate}
\item Construction du polynôme de Lagrange:\\
$P(X)= \displaystyle\sum_{i=1}^{n+1}l_{i}(X)Y_{i}$ avec ${l_{i}}(x)=\displaystyle\prod_{j=1,j\neq i}^{n+1} \frac{X-X_{j}}{X_{i}-X_{j}}$.\\

Calcul des différentes valeurs de Y:\\
$Y_{1}=f(-1)=2$; $Y_{2}=f(X_{2})=-1$; $Y_{3}=f(X_{3})=0$ \\

Cherchons les différentes $l_{i}$:\\
$l_{1}(X)=(\frac{X-X_{2}}{X_{1}-X_{2}})(\frac{X-X_{3}}{X_{1}-X_{3}})$\\
$l_{1}(X)=\frac{X^{2}-X}{2}$ \\ 

$l_{2}(x)=(\frac{X-X_{1}}{X_{2}-X_{1}})(\frac{X-X_{3}}{X_{2}-X_{3}})$\\
$l_{2}(x)= -X^{2}+1$ \\

$l_{3}(x)=(\frac{X-X_{1}}{X_{3}-X_{1}})(\frac{X-X_{2}}{X_{3}-X_{2}})$\\
$l_{3}(x)= (\frac{X+1}{2})(X)=\frac{X^{2}+X}{2}$ \\

Le polynôme de Lagrange est donnée par:\\

\[P(X)=l_{1}(X)Y_{1}+l_{2}(X)Y_{2}+l_{3}(x)Y_{3}\]
$P(X)=2(\frac{X^{2}-X}{2})-(-X^{2}+1)$ \\
$P(X)= X^{2}-X+X^{2}-1 $
  \[\boxed{P(X)=2X^{2}-X-1}\] \\

\item  le polynôme de Newton s'écrit \\ $P(X)=\lambda_{1}+\lambda_{2}(X-X_{1})+\lambda_{3}(X-X_{1})(X-X_{2})$ \\
 
$\lambda_{1}=Y_{1}=2$\\

$\lambda_{2}=Y[X_{1},X_{2}]=\frac{Y[X_{2}]-Y[X_{1}]}{X_{2}-X_{1}}$\\
$\lambda_{2}=\frac{Y_{2}-Y_{1}}{X_{2}-X_{1}}=\frac{-1-2}{0+1}=-3$ \\

$\lambda_{3}=Y[X_{1},X_{2},X_{3}]$\\

$\lambda_{3}=\frac{Y[X_{2},X_{3}]-Y[X_{1},X_{2}]}{X_{3},X_{1}}$\\
 
$\lambda_{3}=\frac{\frac{Y_{3}-Y_{2}}{X_{3}-X_{2}}-\frac{Y_{2}-Y_{1}}{X_{2}-X_{1}}}{X_{3}-X_{1}}$\\

$\lambda_{3}=\frac{\frac{0+1}{1-0}-\frac{-3}{0+1}}{1+1}=2$\\ 

Le polynôme de Newton est donné par: \\
$P(X)=\lambda_{1}+\lambda_{2}(X-X_{1})+\lambda_{3}(X-X_{1})(X-X_{2})$
$P(X)=2-3(X+1)+2(X+1)(X)$ \\ 
\[\boxed{P(X)=2X^{2}-X-1}\]
\item La droite de régression au sens des moindres carrés des points $(-1,2), (0,-1),(1,0)$\\

Une équation de la droite de régression par la méthodes moindres carrés est: \[y=ax+b\] Avec $a=\frac{\displaystyle\sum_{i=0}^n (X_{i}-\overline{X})(Y_{i}-\overline{Y})}{\displaystyle\sum_{i=0}^n (X_{i}-\overline{X})^{2}}$ et $b=\overline{Y}-a\overline{X}$\\

Calcul des moyennes $\overline{X}$ et $\overline{Y}$:\\

$\overline{X}=\frac{1}{3}(-1+0+1)=0$\\$\overline{Y}=\frac{1}{3}(+2-1+0)=\frac{1}{3}$\\

$a=\frac{(X_{1}-\overline{X})(Y_{1}-\overline{Y})+(X_{2}-\overline{X})(Y_{2}-\overline{Y})+(X_{3}-\overline{X})(Y_{3}-\overline{Y})}{(X_{1}-\overline{X})^{2}+(X_{2}-\overline{X})^{2}+(X_{3}-\overline{X})^{2}}$

$a=\frac{(-1)(2-\frac{1}{3})+(0)(-1-\frac{1}{3})+(1)(-\frac{1}{3})}{(-1)^{2}+(0)^{2}+(1)^{2}}=-1$ \\
On trouve:
$\boxed{a=-1}$ et $\boxed{b=\frac{1}{3}}$ \\ Ainsi l'équation est donnée par: $\boxed{y=-x+\frac{1}{3}}$

\end{enumerate}

\section{Exercice 2}


\end{document}